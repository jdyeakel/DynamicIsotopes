\documentclass[12pt]{iopart}
\usepackage{amsmath}
\usepackage{amssymb}
\usepackage{cite}
\begin{document}
\title{Isotopic incorporation and the temporal dynamics of foraging among individuals and within populations}
\begin{abstract}
\end{abstract}
\maketitle

\section{Introduction}

\section{Time-dependent Ornstein-Uhlenbeck process}
We can define the evolution of the concentration of a particular isotope in the predator tissue by the following Langevin equation,
\begin{equation}
\label{langevin}
\mathrm{d}X(t) = - \lambda\left(X(t)-\mu(t)\right)\mathrm{d}t + \lambda\sigma(t)\mathrm{d}W
\end{equation}
where, $X$ is the isotope-concentration, $\lambda$ is the tissue-dependent incorporation rate, $\mu(t)$ and $\sigma(t)$ are the time-dependent average and standard-deviations of isotope-concentration over the prey.
\subsection{Expectation}
 We can find the expectation of $X(t)$ as follows,
\begin{equation}
\label{EX1}
\mathrm{d}X + \lambda\left(X(t)-\mu(t)\right)\mathrm{d}t = \lambda\sigma(t)\mathrm{d}W
\end{equation}
Multiplying throughout by the integrating factor, $e^{\lambda t}$ and integrating both sides w.r.t $t$, we get
\begin{equation}
\label{EX2}
\int_{t=0}^t \mathrm{d}\left(e^{\lambda t}\left(X-\mu)\right)\right) + \int_{t=0}^t e^{\lambda t'}\frac{\mathrm{d}\mu}{\mathrm{d}t'}\mathrm{d}t' = \int_{t=0}^t\lambda e^{\lambda t'}\sigma(t')\mathrm{d}W'
\end{equation}
Taking the expectation of both sides, the RHS reduces to zero as $\mathbb{E}(\mathrm{d}W)=0$ and we get,
\begin{equation}
\label{EX-final}
\mathbb{E}(X(t)) = X_0 e^{-\lambda t} + \lambda e^{-\lambda t}\int_{t'=0}^t\mu(t')e^{\lambda t'}\mathrm{d}t'
\end{equation}

\subsection{Variance}
Similarly for the variance, we square Eq.~\eqref{EX2}, to get
\begin{equation}
\label{VX1}
\left(e^{\lambda t}X - X_0 - \lambda\int_0^t \mu e^{\lambda t'}\mathrm{d}t'\right)^2=\lambda^2\int_0^t\int_0^t e^{\lambda(t'+t'')}\sigma(t')\sigma(t'')\mathrm{d}W(t')\mathrm{d}W(t'')
\end{equation}
Taking the expectation on both sides, in the RHS, using the property, $\mathbb{E}(\mathrm{d}W(t')\mathrm{d}W(t'')) = \delta(t'-t'')\mathrm{d}t'\mathrm{d}t''$, we get,
\begin{equation}
\label{VX2}
\mathbb{E}\left(e^{\lambda t}X - X_0 - \lambda\int_0^t \mu e^{\lambda t'}\mathrm{d}t'\right)^2=\lambda^2\int_0^t e^{2\lambda t'}(\sigma(t'))^2\mathrm{d}t'
\end{equation}
Let us call the integrals $\int_0^t\mu(t') e^{\lambda t'}\mathrm{d}t'$ and $\int_0^t e^{2\lambda t'}(\sigma(t'))^2\mathrm{d}t'$ as $\mathcal{U}$ and $\mathcal{S}^2$ and expanding the LHS, we get
\begin{equation}
\label{VX3}
e^{2\lambda t}\mathbb{E}(X^2) + X_0^2 + \lambda^2 \mathcal{U}^2 - 2 X_0 \mathbb{E}(X) e^{\lambda t} - 2 X_0 \lambda\mathcal{U}-2\lambda e^{\lambda t}\mathbb{E}(X)\mathcal{U} = \lambda^2\mathcal{S}^2
\end{equation}
And so the variance is given by,
\begin{align}
\label{VX4}
\mathbb{V}(X) &= \mathbb{E}(X^2)-(\mathbb{E}(X))^2 \nonumber \\
&= \lambda^2 e^{-2\lambda t}\mathcal{S}^2 - X_0^2 e^{-2\lambda t} - \lambda^2 e^{-2\lambda t}\mathcal{U}^2 + 2 X_0 \mathbb{E}(X)e^{-\lambda t} +2 X_0 \lambda e^{-2\lambda t}\mathcal{U}+2\lambda e^{-\lambda t}\mathbb{E}(X)\mathcal{U} - (\mathbb{E}(X))^2
\end{align}
\begin{equation}
\label{VX-final}
\mathbb{V}(X) = \lambda^2 e^{-2\lambda t} \int_0^t e^{2\lambda t'}(\sigma(t'))^2\mathrm{d}t'
\end{equation}
where, to go from Eq.~\eqref{VX4} to Eq.~\eqref{VX-final} we used $\mathbb{E}(X)$ given by Eq.~\eqref{EX-final}.

\subsection{Sinusoidal input}
The previous section gives the time dependence of the expected value and the variance for arbitrary input functions of the prey-isotope values. To get a better insight into the formula, let us study a simpler case. Our aim is to understand how a periodic prey-switching behavior affects the expectation and variance of the consumer isotope values. The simplest oscillatory behavior is when the input functions, i.e., the prey-isotope expectation and variances vary sinusoidally with time.
Suppose the inputs $\mu(t)$ and $\sigma(t)$ to the system are sinusoidal, i.e,
\begin{equation}
\label{sin-input}
\mu(t)=a_1 + b_1 \sin{(\omega t)}\;,\quad\sigma(t)=a_2+b_2\sin{(\omega t)}
\end{equation}

 The offsets, $a_1$ and $a_2$ are the average of the mean and variances of the two different sets of prey it is oscillating between, respectively. The amplitudes, $b_1$ and $b_2$ are half the differences between the mean and variances of the two sets of prey, respectively. 
Using Eq.~\eqref{EX-final} and Eq.~\eqref{VX-final}, we get
\begin{equation}
\label{EX-sin-full}
\mathbb{E}(X) = X_0 e^{-\lambda t} + a_1\left(1-e^{-\lambda t}\right) + \frac{b_1\omega\lambda}{\lambda^2+\omega^2} e^{-\lambda t} + \frac{b_1 \lambda}{\sqrt{\lambda^2+\omega^2}} \sin{\left(\omega t + \tan^{-1}{\left(\frac{\omega}{\lambda}\right)}\right)}
\end{equation}
\begin{align}
\label{VX-sin-full}
\mathbb{V}(X) &= \frac{1}{4}\lambda\left(2 a_2^2 + b_2^2\right)\left(1-e^{-2\lambda t}\right) +e^{-2\lambda t}\lambda^2\left(\frac{b_2^2\lambda}{4(\lambda^2+\omega^2)}+\frac{2 a_2 b_2 \omega}{4\lambda^2 + \omega^2}\right) \nonumber \\
&+ \frac{2 a_2 b_2 \lambda^2}{\sqrt{4\lambda^2+\omega^2}} \sin{\left(\omega t + \tan^{-1}{\left(\frac{\omega}{2\lambda}\right)}\right)}-\frac{b_2^2\lambda^2}{4\sqrt{\lambda^2+\omega^2}} \sin{\left(2 \omega t + \tan^{-1}{\left(\frac{\lambda}{\omega}\right)}\right)}
\end{align}
Getting rid of transients, we get,
\begin{equation}
\label{EX-sin}
\mathbb{E}(X) = a_1 + \frac{b_1 \lambda}{\sqrt{\lambda^2+\omega^2}} \sin{\left(\omega t + \tan^{-1}{\left(\frac{\omega}{\lambda}\right)}\right)}
\end{equation}
\begin{align}
\label{VX-sin}
\mathbb{V}(X) = \frac{1}{4}\lambda\left(2 a_2^2 + b_2^2\right) &+ \frac{2 a_2 b_2 \lambda^2}{\sqrt{4\lambda^2+\omega^2}} \sin{\left(\omega t + \tan^{-1}{\left(\frac{\omega}{2\lambda}\right)}\right)} \nonumber \\ &- \frac{b_2^2\lambda^2}{4\sqrt{\lambda^2+\omega^2}} \sin{\left(2 \omega t + \tan^{-1}{\left(\frac{\lambda}{\omega}\right)}\right)}
\end{align}
If we time-average over the oscillations, we get,
\begin{equation}
\label{EX-VX-sin-timeaverage}
\left\langle \mathbb{E}(X)\right\rangle_t = a_1\;,\quad \left\langle\mathbb{V}(X)\right\rangle_t = \frac{1}{4}\lambda\left(2 a_2^2+b_2^2\right)
\end{equation}

\section{On the input to the O-U equation}
In the previous section we had assumed $\mu(t)$ and $\sigma(t)$ was given by some underlying dynamics. One possible way to incorporate the dynamics is to assume a Lotka-Volterra system. The simplest system where we can expect prey switching behavior is a Lotka-Volterra system with one predator and two preys. We have,
\begin{align}
\label{Lotka-Volterra-system}
\dot{c} &= -r_0 c + a c n_1 + a c n_2 \nonumber \\
\dot{n}_1 &= r_1 n_1 - c n_1 \nonumber \\
\dot{n}_2 &= r_2 n_2 - c n_2
\end{align}
where $c$ is the number density of the consumer, $n_i$ are the number densities of the prey. The dynamics we adopt are deterministic. However, we introduce noise in the encounter rates as follows. We define the encounters as a Poisson process with encounter rate given by a gamma distribution with parameters $(n_i/v, 1/v)$. Here $v$ is the `patchiness' or the additional variance to constant rate Poisson process. For simplicity we assume the patchiness is the same accross the preys. Imposing constant feeding rate, we combine the two independent gamma distribution to a two variable Dirichlet distribution with parameters $(n_1/v, n_2/v)$.


\section{Bayesian analysis}
Until this point, we have discussed the forward problem, i.e., given the parameters of the system, what are the possible outcomes of the process. Now we would like to answer the question, given the measurements of the outcomes, what is the best estimate or the distribution of estimates of the parameters of the system. Is there a threshold for the noise in the system above which it is impossible to conclude if there is switching in the system or not. We first begin with the simpler problem of only the amplitude of oscillations being unknown. Let us assume that the periodicity of the oscillation is tied to the seasonal cycle and hence one year.
We can write the



\end{document}
