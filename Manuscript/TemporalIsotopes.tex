\documentclass[11pt]{article}

\usepackage[margin=1in,footskip=0.25in]{geometry}

%\usepackage{helvet}
%\renewcommand{\familydefault}{\sfdefault}

\renewcommand\refname{\vskip -1cm}

%\renewcommand{\rmdefault}{phv} % Arial
%\renewcommand{\sfdefault}{phv} % Arial
\usepackage{setspace}
\usepackage{wrapfig}
\usepackage{amsmath}
\usepackage{amssymb}
\usepackage{graphicx}
\usepackage{mathrsfs}
\usepackage{bm}
\usepackage{wasysym}
\usepackage{placeins}
\usepackage{multirow}
\usepackage[T1]{fontenc}
\usepackage[super]{natbib}
\usepackage{framed}
\usepackage{caption}
\usepackage{longtable}

\begin{document}

\title{Isotopic incorporation and the temporal dynamics of foraging among individuals and within populations}
\author{JD Yeakel, U Bhatt, SD Newsome}
\maketitle

\section{Introduction}

Consumer foraging behaviors are dynamic, resulting in diets that change over time as a function of environmental conditions, the densities of consumer and resource populations, and even the physiological states of individual foragers.
Understanding how diets change, and to what extent different conditions promote or inhibit specific changes, is both a challenging theoretical and empirical problem in ecology. 

%Difference between the isotopic diet view vs. other diet views




%Prey-switching dynamic
%Micro time frame
%Macro time frame




\section{Methods}

\section{Results}

\section{Discussion}






\end{document}
